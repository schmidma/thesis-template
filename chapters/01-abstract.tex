\pdfbookmark{Abstract / Kurzfassung}{Abstract}
\section*{Abstract}

Latex is a natural polymer that is widely used in the production of a variety of products, including gloves, balloons, condoms, and medical devices.
This comprehensive review provides a detailed overview of the chemistry and properties of latex, including its composition, structure, and processing methods.
The review begins with a discussion of the chemical constituents of latex, including proteins, carbohydrates, lipids, and other minor components.
The structure and properties of the latex particle are then examined, including its size, shape, and surface properties.
The review also covers the various processing methods used to produce latex products, such as dipping, casting, and foaming.
In addition to its composition and processing methods, the review also examines the physical and mechanical properties of latex, including its elasticity, strength, and tear resistance.
The review also explores the thermal and chemical stability of latex, as well as its biocompatibility and potential allergenicity.
The environmental impact of latex production and disposal is also discussed.
Overall, this review provides a comprehensive overview of the chemistry and properties of latex, highlighting its importance as a versatile and widely used material.
The review also identifies areas where further research is needed, such as in the development of sustainable and environmentally friendly production methods, as well as in the identification and mitigation of potential health hazards associated with the use of latex products.

\vfill

\section*{Kurzfassung}

Latex ist ein natürlicher Polymer, der in der Produktion einer Vielzahl von Produkten, einschließlich Handschuhe, Ballons, Kondome und medizinische Geräte, weit verbreitet ist.
Diese umfassende Übersicht bietet einen detaillierten Überblick über die Chemie und Eigenschaften von Latex, einschließlich seiner Zusammensetzung, Struktur und Verarbeitungsmethoden.
Die Übersicht beginnt mit einer Diskussion der chemischen Bestandteile von Latex, einschließlich Proteinen, Kohlenhydraten, Lipiden und anderen geringfügigen Komponenten.
Die Struktur und Eigenschaften des Latexpartikels werden dann untersucht, einschließlich seiner Größe, Form und Oberflächeneigenschaften.
Die Übersicht behandelt auch die verschiedenen Verarbeitungsmethoden, die zur Herstellung von Latexprodukten verwendet werden, wie Tauchen, Gießen und Schaumstoff.
Neben seiner Zusammensetzung und Verarbeitungsmethoden untersucht die Übersicht auch die physikalischen und mechanischen Eigenschaften von Latex, einschließlich seiner Elastizität, Festigkeit und Reißfestigkeit.
Die Übersicht untersucht auch die thermische und chemische Stabilität von Latex sowie seine Biokompatibilität und potenzielle Allergenität.
Die Umweltauswirkungen der Latexproduktion und -entsorgung werden ebenfalls diskutiert.
Insgesamt bietet diese Übersicht einen umfassenden Überblick über die Chemie und Eigenschaften von Latex und hebt dessen Bedeutung als vielseitiges und weit verbreitetes Material hervor.
Die Übersicht identifiziert auch Bereiche, in denen weitere Forschung erforderlich ist, wie die Entwicklung nachhaltiger und umweltfreundlicher Produktionsmethoden sowie die Identifizierung und Minderung potenzieller Gesundheitsgefahren im Zusammenhang mit der Verwendung von Latexprodukten.



\vfill
